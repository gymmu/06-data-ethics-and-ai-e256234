\documentclass{article}

\usepackage[ngerman]{babel}
\usepackage[utf8]{inputenc}
\usepackage[T1]{fontenc}
\usepackage{hyperref}
\usepackage{csquotes}

\usepackage[
    backend=biber,
    style=apa,
    sortlocale=de_DE,
    natbib=true,
    url=false,
    doi=false,
    sortcites=true,
    sorting=nyt,
    isbn=false,
    hyperref=true,
    backref=false,
    giveninits=false,
    eprint=false]{biblatex}
\addbibresource{../references/bibliography.bib}

\title{Review des Papers "Ethik im Umgang mit Daten" von Sara Broch Jahandar\dots}
\author{Zoe Donghi}
\date{\today}

\begin{document}
\maketitle

\abstract{
    Dies ist ein Review der Arbeit zum Thema Ethik im Umgang mit Daten von Sara Broch Jahandar.

}

\section{Review}
Persönlich gefällt mir die Arbeit sehr. Die Art und Weise wie der Text geschrieben und formuliert ist, macht es für den Leser spannend den Text zu lesen und auch ein Interesse für das Thema zu wecken.
Ich finde das Thema wurde in dem Text sehr deutlich beschrieben und auch mit Beispielen untermauert. 
Einerseits gefällt mir, dass der Text kurz gefasst und trotzdem sehr informativ ist, dennoch denke ich hätte man noch mehr aus dem Thema herausholen können.
Bei einem längeren Text hätte man noch mehr ins Detail gehen können oder auch noch mehr einzelne Informationen über verschiedene Beispiele finden und nennen können.
So hätten die Informationen über den gesamten Text verteilt werden können und auch genauer beschrieben werden können, was die Lesbarkeit gesteigert hätte. 
Im Gesamten fnde ich wird das Thema in dem Text so erklärt, dass man einen Einblick in das Thema und die verschiedenen Folgen oder auch Vorteile, welche KI mit sich bringt, informiert.


\printbibliography

\end{document}

