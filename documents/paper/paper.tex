\documentclass{report}

\usepackage[ngerman]{babel}
\usepackage[utf8]{inputenc}
\usepackage[T1]{fontenc}
\usepackage{hyperref}
\usepackage{csquotes}
\usepackage[a4paper]{geometry}

\usepackage[
    backend=biber,
    style=apa,
    sortlocale=de_DE,
    natbib=true,
    url=false,
    doi=false,
    sortcites=true,
    sorting=nyt,
    isbn=false,
    hyperref=true,
    backref=false,
    giveninits=false,
    eprint=false]{biblatex}
\addbibresource{../references/bibliography.bib}


\title{Ethik im Umgang mit Daten}
\author{Zoe Donghi}
\date{\today}


\begin{document}

\maketitle



\tableofcontents

\chapter{Einleitung}

KI empfängt über den Computer Daten, welche er dann verarbeitet und den Menschen dadurch hilft verschiedene Probleme zu lösen oder auch auf Fagen eine Antwort geben zu können.
KI ist ein wesentlicher Treiber für die Transformation in unserer Gesellschaft.

\section{Was ist KI und ihr bezug zur Ethik?}
KI ist ein Teilgebiet der Informatik. Es beschränkt sich auf maschinelles Lernen.
Der Bedarf an Regeln für eine Verantwortungsvolle Handhabung steigt, durch die immer größer werdende Nutzung der immer stärkeren leistungsstarken datenbasierten Systeme. 
Ethik und Technologie sollen miteinander in Interaktion treten um sich gegenseitig bereichern zu können. Dadurch können auch ethische Chancen und Risiken frühzeitig erkannt und behoben werden. 
KI bezeichnet die Fähigkeit von Maschinen und Computersysteme, Aufgaben auszuführen welche normalerweise menschliche Intelligenz erfordern. 
Die Frage ob KI bereits ethisch vertretbare Entscheidungen treffen kann ist unter Experten umstritten. 

\section{Wo wird KI genutzt?}
\begin{itemize}
\item Sprachassistenten wie Siri, Cortana und Alexa 
\item Soziale Medien: Seiten wie Instagram, Facebook, Pinterest und auch Twitter nutzen KI um die Inhalte zu personalisieren und Genau die Inhalte zu zeigein die der Nutzer sehen will.
\item Online- Suche: Suchen wir nach einem bestimmten Thema werden un auf Googel und co. direkte antworten geliefert. Auch dies Funktioniert durch Algorithmen.
\item Finanzwesen: Im Finanzwesen wird KI unter anderem dafür genutzt um Betrugsversuche zu verhindern.
\item Autonome Fahrzeuge:Wie alles andere werden auch die Fahrzeuge moderner. Dafür nutzt man KI um verschiedene Sensoren so zu programieren das sie auf Distanzen und co. aufmerksam werden und so ein sichereres fahren ermöglichen. 
\item Übersetzungsdienste:Übersetzer wie Googel-Translate nutzen KI um Texte autoatisch zu übersetzen. 
\item Gesichtserkennung: KI wird auch zu der Gesichtserkennungnbei Identifikationsdiensten in den sozialen Medien genutzt. 
\item Online-Shopping: Ki wird im Online-Shopping ebenfalls genutzt, in Form davon , dass KI Kundenfragen beantworten kann, den Bestellstatus verfolgen kann aber auch sonst Probleme lösen kann. 
\item Medizinische Bildgebung. In der Medizin wird KI als Hilfe für zum Beispiel Röntgenbilder verwendet. 
\end{itemize}

\section{Wie funktioniert KI?}
KI kann menschliche Fähigkeiten durch Eingabedaten, aus welchen sie Informationen nimmt und sortiert, immitierten

\section{Was sind die Grenzen von KI und welche Risiken birgt KI?}
Die Grenze von KI besteht zum Beispiel darin, dass egal welches KI- Tool man einsetzt keines in der Lage sein wird ein Unternehmen selbst zu leiten oder grundsätzlich eine führende Position zu besitzen.
Eine der grössten Risiken die KI birgt ist der Datenschutz. KI arbeitet mit einer grossen Menge an Daten, welche verarbeitet werden und aus welchen auch Informationen genommen werden. 
Sind die Daten also nicht richtig verchlüsselt oder geschützt kann es zu einem Datennschutzmissbrauch kommen. 
ein weiteres Risiko, welches Ki betrifft ist die eigentliche Nutzung. KI lässt sich nämlich auch dazu verwenden Menschen zu manipulieren oder zu täuschen. 
so kann es sein das die eigene Meinung durch KI beeinflusst oder auch verändert wird. 
\input{chap_methode.tex}
Neutronale KI vs. Symbolische KI  und Simulationsmethodebasierte KI vs.Phänomenologiemathodenbasierte KI sind verschiedene Methoden von KI welche miteinander zusammenhängen. 
Die verschiedenen Methoden verfolgen verschiedene Ansätze. Die Neuronale KI versucht hauptsächlich die Intelligenz des menschlichen Gehirns so gut wie möglich nachzustellen während das Ziel der Symbolischen Ki bBegrifflichen Ebene der intelligenz bleibt. 
Die Simulationsmethodebasierte KI versucht die kognitiven Prozesse der Menschen so gut wie mölich zu imitieren. Bei der Phänomenologiemethodenbasierten KI hingegen kommt es nur auf das Ergebnis an. 
Diese verschiedene Methoden stehen miteinander in Verbindung und haben auch einen gewissen Einfluss aufeinander. 


\section{Konkrete Methoden}
\begin{itemize}
\item Natural Language Processing
\item Computer Vision
\item Anomaly Detection
\item Decision Tree 
\item k-Means
\end{itemize}

\nocite{*}
\printbibliography

\end{document}
